%%%%%%%%%%%%%%%%%%%%%%%%%%%%%%%%%%%%%%
%     FLAGS
%%%%%%%%%%%%%%%%%%%%%%%%%%%%%%%%%%%%%%

% Canadian Government-specified colors for the flag
\definecolor{cared}{RGB}{239,51,64}
\definecolor{cawhite}{RGB}{255,255,255}

% Other government specifications include:
% Each red portion on the left/right is 1/4 flag width, here it is 48cm

%Source: https://en.wikipedia.org/wiki/Flag_of_Canada#/media/File:Flag_of_Canada_(construction_sheet_-_leaf_geometry).svg

%Created by Senan Sekhon, August 5, 2021

\begin{center} %Optional, but helps to tidy up the layout
\begin{tikzpicture}[scale=1] %Scale must be changed to make the flag fit on letter/A4 paper (scale=1 produces a 192 cm by 96 cm flag)
\clip (-96,-48) rectangle (96,48); %Optional, crops the flag to the correct size
\draw[-] (-96,-48) rectangle (96,48); %Optional, draws a border around the flag
\fill[cared] (-96,-48) rectangle (-48,48); %Red portion on the left
\fill[cared] (48,-48) rectangle (96,48); %Red portion on the right
\fill[cawhite] (-48,-48) rectangle (48,48); %White portion in the center

% Specifying the coordinates of the sharp corners of the maple leaf
\coordinate (A) at (-1.8,-40.6);
\coordinate (B) at (-20.3,-24.4);
\coordinate (C) at (-37.2,-1.3);
\coordinate (D) at (-36,13.7);
\coordinate (E) at (-21.6,17.1);
\coordinate (F) at (-15,30.2);
\coordinate (G) at (0,40);
\coordinate (H) at (15,30.2);
\coordinate (I) at (21.6,17.1);
\coordinate (J) at (36,13.7);
\coordinate (K) at (37.2,-1.3);
\coordinate (L) at (20.3,-24.4);
\coordinate (M) at (1.8,-40.6);

\foreach \p/\q/\tp/\tq/\r in {% Specifying the coordinates of the rounded corners of the maple leaf
A/B/87/10/1.9,
B/C/70/321/1.3,
C/D/25/288/1.3,
D/E/348/247/1.3,
E/F/313/281/1.3,
F/G/330/243/1.3,
G/H/297/210/1.3,
H/I/259/227/1.3,
I/J/293/192/1.3,
J/K/252/155/1.3,
K/L/219/110/1.3,
L/M/170/93/1.9}
{% Phantom lines from each point, length set to 30cm to ensure they intersect but stay within the flag boundary
\path[name path=line \p1] (\p)--++(\tp:30);
\path[name path=line \p2] (\q)--++(\tq:30);
% Intersection point of phantom lines
% Adapted from https://tex.stackexchange.com/questions/31398/tikz-intersection-of-two-lines
\path[name intersections={of=line \p1 and line \p2,by=\p\q}];
% Offset length to start of arc
\pgfmathsetmacro{\ds}{\r/tan((\tq-\tp)/2)}
% Coordinate of start of arc
\coordinate (\p\q0) at ([shift={(\tp:\ds)}] \p\q);}
% Maple leaf
\fill[cared] (A)--(AB0) arc(-3:100:1.9)--(B)--(BC0) arc(-20:51:1.3)--(C)--(CD0) arc(-65:18:1.3)--(D)--(DE0) arc(258:337:1.3)--(E)--(EF0) arc(223:371:1.3)--(F)--(FG0) arc(240:333:1.3)--(G)--(GH0) arc(207:300:1.3)--(H)--(HI0) arc(169:317:1.3)--(I)--(IJ0) arc(203:282:1.3)--(J)--(JK0) arc(162:245:1.3)--(K)--(KL0) arc(129:200:1.3)--(L)--(LM0) arc(80:183:1.9)--(M)--cycle;
\end{tikzpicture}
\end{center}
